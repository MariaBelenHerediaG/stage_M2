\documentclass{article}

%Packages R
\usepackage{url}
\usepackage{amssymb}

\begin{document}

%13/02/2016
%Tareas:
% Leer paper
% Reuni�n Cl�mentine Prieur.
% Cours WRF.
% Solicitar Datos a Thomas.
% Inventario de datos !2011-2015
% Analizar variable de precipitaci�n a varios niveles de agregaci�n simulacion-observacion in situ.
% C�mo determinar anios secos y h�medos? Encontrar una tendencia?
% Leer kriging
% Revisar scripts de Lise para aplicar kriging.
We have a WRF simulation of a period of time of four years (3 domains: 27 km (d01), 9 km (d02), 3 km (d03), 1 km (d04)) with the same parametrization as simu1 and the aditional options: MI = 2 (parameter concerning to the microphysics options in WRF), slope\_rand=1 (change in the angle of the sun incidence) and topo\_shading=1 (add shadow effect to a mountain).   

$MI=2$ \cite{Lang1983}:
\begin{itemize}
\item 5-class microphysics including graupel\footnote{Graupel is precipitation that forms when supercooled droplets of water are collected and freeze on falling snowflakes obtained from \url{https://en.wikipedia.org/wiki/Graupel}}.
\item Include ice sedimenation and time-split fall terms.
\end{itemize}


\section{NCL}
Commands:
\begin{itemize}
\item \texttt{ncrcat fichier1.nc fichier2.nc  fichier\_result.nc}: Concatenar dos ficheros.
\item \texttt{ncrcat *.nc fichier\_result.nc}: Concatenar todos los ficheros de extension nc.
\end{itemize}

\section{Papers}
\subsection{Spatio-temporal metamodeling for West African monsoon}
? Como determinar que el cambio en el albedo es influente para el Monsoon.\\
? Grid-computing environment. \\
? Functional principal component approach as a dimensional reduction technique.\\
? Double penalized regression approach.\\
? Functional data analysis (FDA) (Introduction: Ramsay and Silverman 2002 and 2005).\\
In this work the authors propose a new approach for modeling and fitting high-dimensional response regressions models in the setting of complex spatio-temporal dynamics. The study is developped for the influence of the sea surface temperatures in the Gulf of Guinea on precipitation in Saharan and sub-Saharan. West African monsoon a atmospheric phenomeno which causes have not been yet determined in an unequivocal manner but there is a strong evidence which suggests that spatio-temporal changes in the sea surface temperatures in the Gulf od Guinean and changes in the Saharan and sub-Saharan albedo are major factors. The model used for simulating precipitation is MAR. \\
 
\textbf{Global sensitivity} analysis of a model output consists in quantifying the respective importance of input factors over their entire range of values. The problem:  Performing a global sensitivity analysis implies running the model a large number of times. The solution: A way for overcoming this issue is to fit a stochastic
model, which approximates the MAR by taking into consideration the spatio-temporal dynamic of the underlying physical phenomenon and with the ability to be run in a reasonable time. They focus on the regression of precipitation on sea surface temperatures. \\

The data: 18 years of data 1983 to 2000 of sea surface temperature, and 8 years for precipitation 1983-1990. \\
$X^{x}:$ Sea surface.\\
$Y^{y}:$ Precipitation.\\

Metamodel constructions steps: dimension reduction and regression. First, funcional principal component approach is developped over the two variables (precipitation and sea surface temperature) is performed. Then, the Karhunen-Lo\'eve decomposition is then truncated because major part of the variance is explained by only few terms. Second, a functional clustering algorithm is performed on the selected eigenfunctions to reduce the spatial dispersion of the Karhunen-Lo\'eve eigenfunctions. Third, the relationship between the outputs and inputs is modeled on the coefficients of the decomposition through a double penalized regression approach.\\

\textbf{Data description:} SST of 516 points of a spatial grid $G$ located in the Gulf of Guinea, weekly data from May to November. 18 years of observations from 1983 to 2000. Precipitation of 368 points of a spatial grid $G'$, daily data whose mean is computed on 10 consecutive days form March to November from 1983 to 1990.

\textbf{Modeling inputs and outputs:} Description of the methodology to study the relationship between precipitation and the SST. \\
Let $\tau$ be a finite and closed interval of $\mathbb{R}$. The spatial regions of interest are $R$ and $R'$. Let $x$ be a grid point on $G$. They considered  that the $i$th observed time-dependent trajectory at point $x$ correspond to a sampled longitudinal curve viewed as a realizations of random trajectories $(X_{i}^{x}), i=1,...N$. These $X_{i}^{x}$ were viewed as independent realizations of the stochastic process $X^{x}$ which mean function $E(X^{x})(t)=\mu_{X^{x}}(t)$ is unkown, and covariance function is $cov(X^{x}(s), X^{x}(t))=G_{X}^{x}(s,t)$. Then, there exist a decomposition of $G_{X^{x}}$ in the $\mathbb{L^{2}}$ sense in terms of eigenfunctions $e_{m}(x,.)$ with associeted eigenvalues $\rho_{m}(x)$:

\begin{equation*}
G_{X^{x}(s,t)}=\sum_{m \geq 1} \pho_{m}(x)e_{m}(x,s)e_{m}(x, t), \qquad s, t \in \tau.
\end{equation*}

The random function $X^{x}(t)$ where $t$ denotes time and $x$ location, may be decomposed into orthogonal expansion:

\begin{equation*}
X^{x}(t)=\mu_{X^{x}}(t)
\end{equation*}

\bibliographystyle{alpha}
\bibliography{biblio.bib}
\end{document}